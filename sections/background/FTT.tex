%! Author = mbaddar
%! Date = 4/25/24

% Preamble
\documentclass[11pt]{article}

% Packages
:.
% Document
\begin{document}
    Let $\mathbf{y}=\mathbf{h}(\mathbf{x}) $ be a vector-valued function where \cite{Calculus40:online}  $ \mathbf{h}:\mathbb{R}^{D_x} \rightarrow \mathbb{R}^{D_y}$.
    A tensor-train-based ansatz $\mathbf{f} \approx \mathbf{h}$ can be formulated as:
    \begin{equation}
        \label{eq:ftt-overall-formulation}
        (y_1,y_2,\dots,y_{j},y_{D_y}) = \mathbf{f}(\mathbf{x}) = (f_1(\mathbf{x}),f_2(\mathbf{x}),\dots,f_j(\mathbf{x}),\dots,f_{D_y}(\mathbf{x}))
    \end{equation}
    And for each $y_j \quad j=1,2,\dots,D_y$, the component function \cite{Calculus40:online}
    $f_j(\mathbf{x})=f_j(x_1,\dots,x_i,\dots,x_{D_x}): \mathbb{R}^{D_x} \rightarrow \mathbb{R} $
    can be modeled as a FTT of order $D_x$, which can be formulated as:


    \begin{subequations}
        \label{eq:ftt}
        \begin{align}
            f(\mathbf{x}) &= \bm{A}\Phi(\mathbf{x}) \label{ftt1}\\
            \bm{A}[\bm{k}] &= \sum_{\bm{s}}^{\bm{r}} G_1[s_0,k_1, s_1]\dots G_i[s_{i-1},k_i, s_i] \dots G_{D_x}[s_{D_x-1},k_D, s_{D_x}]\\
            \bm{\Phi}(\mathbf{x})&=\bigotimes_{i=1}^{D_x} \bm{\phi}_i(x_i)
        \end{align}
    \end{subequations}
    Where
    \begin{align*}
        \mathbf{A},\bm{\Phi} &\in \mathbb{R}^{m_1 \times \dots \times m_{i}\times \dots \times m_{D_x}} \\
        i&=1,2,\dots,D \textit{  Index over Tensor-Train dimensions}\\
        \bm{k}&=(k_1,k_2,\dots,k_i,\dots,k_{D_x}) \quad k_i=1,2,\dots,m_i \textit{  Index over Tensor-Train elements}\\
        \bm{r}&=(r_0,r_1,\dots,r_i,\dots,r_{D_x}) \textit{  Tensor train operator ranks}\\
        r_{0}&=r_{D_x}=1 \\
        \bm{s}&=(s_0,s_1,\dots,s_i,\dots,s_{D_x}) \textit{  Index over the ranks}\\
        G_i &\in \mathbb{R}^{r_{i-1} \times m_i \times r_i}\\
    \end{align*}
    And the tensor-valued basis function $\Phi(\mathbf{x})$ is modeled by dimensional decomposition as the tensor-product
    of a set of rank-1 basis functions
    \begin{equation}
        \label{eq:ftt-basis-fn-decomp}
        \Phi(\mathbf{x}) = \bigotimes_{i=1}^{D_x} \bm{\phi}_i(x_i)
    \end{equation}
    Where $\bm{x}=[x_i]_{i=1}^{D_x}$ and $\bm{\phi}_i(x_i) : \mathbb{R}\rightarrow\mathbb{R}^{m_i}$.
    \par
    Finally, the contraction algorithm in (\ref{eq:ftt}a) is:
    \begin{enumerate}
        \item Calculate $C_i(x_i) = \sum_{k_i=1}^{m_i} G_i(s_{i-1},k_i,s_i) \bm{\phi}(x_i)$ for all $i=1,2\dots,D$.
        Note that  $C_i \in \mathbb{R}^{r_{i-1} \times r_i}$
        \item $f(\mathbf{x}) = \sum_{i=1}^{D}\sum_{s_{i-1},s_i}^{r_{i-1},r_i} C(x_i)$
    \end{enumerate}
\end{document}