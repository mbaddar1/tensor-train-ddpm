%! Author = mbaddar
%! Date = 4/23/24

% Preamble
\documentclass[11pt]{article}
\title{Tensor-Train Diffusion Models }
\author{M. Baddar , M. Eigel}
\date{\today}
% Packages
\usepackage{natbib}
\usepackage{amsmath}
\usepackage{amsfonts}
\usepackage{xcolor}
\usepackage{comment}
\usepackage{bm}
\usepackage{amssymb}
\usepackage{subfiles}
\usepackage{hyperref}
\usepackage[mathscr]{euscript}
\usepackage[title]{appendix}
\bibliographystyle{alpha}
% Martin comment
\begin{comment}
    https://uq-berlin.slack.com/archives/D0168AT80RY/p1713873401234819
    I had a look at the document and would give you some impressions just as they are crossing my mind:
    - It's good that you try to provide an overview of the foundations of generative models, in particular diffusion and then also tensor networks. This is a central initial task.
    - However, there is lots of text and topics and it appears that it is not focused on what you want to achieve. Moreover, while there are more mathematical details, the topics are only described in a rather shallow overview style.
    - I have doubts about the idea to combine TN and NN since then two functionals have to be optimised, introducing additional complexity and parameters for training.
    - If the plan is to use TTs in DDPM (which is absolutely worth a shot), this has to be described (in a 5-10 pages research plan, not an overview document) in full detail but only what is required for the new method. Goals, structure, possible analytical results, algorithms etc. etc.
    - Not sure about the numerical experiments. At first glance, it seems TT didn't perform too well and then running third party code is good to learn but insights are limited imo.
    - My impression is that you would have to spend much more time on learning TT algorithms for optimisation and experiment with that a lot to see opportunities and to alleviate upcoming problems. Things won't just work with existing things. But then, if your research is supposed to be rather practical, a lot of effort has to be invested in the algorithms. You can ask other students from the group, who all do analysis and numerics. If analysis is limited, this would have to be compensated by work on new methods.
    - It is somewhat unclear to me how TTDE can be used with DDPM since it reconstructs a stationary density, not a measure transport. But you might have an idea about that.

    Some paper that you can use as a guide for Martin suggested
    David - GENERATIVE MODELLING WITH TENSOR TRAIN APPROXIMATIONS OF HAMILTON–JACOBI–BELLMAN EQUATIONS
    https://arxiv.org/pdf/2402.15285

    Charles - APPROXIMATING LANGEVIN MONTE CARLO WITH RESNET-LIKE NEURAL NETWORK ARCHITECTURES
    https://arxiv.org/pdf/2311.03242

    Other Analysis Papers
    Convergence of denoising diffusion models under the manifold hypothesis
    https://openreview.net/pdf?id=MhK5aXo3gB
    Convergence Analysis for General Probability Flow ODEs of Diffusion Models in Wasserstein Distances
    https://arxiv.org/pdf/2401.17958
\end{comment}


% Document
\begin{document}
    \maketitle


    \section{Abstract}\label{sec:abstract}
    \subfile{sections/abstract}

    %%%
    \section{Notion and Preliminaries}\label{sec:notioan-and-preliminarities}
    \subfile{sections/notations.tex}
    %%%


    \section{Background}\label{sec:background}

    \subsection{Denoising Diffusion Probabilisitce Models (DDPM)}\label{subsec:ddpm}
    \subfile{sections/background/DDPM.tex}

    \subsection{Functional Tensor-Trains (FTT)}\label{subsec:functional-tensor-trains}
    \subfile{sections/background/FTT.tex}

    \subsection{Optimization of Function Tensor Trains}\label{subsec:optimization-of-function-tensor-trains}
    \subfile{sections/background/ftt_opt.tex}

    %%%


    \section{Model}\label{sec:proposed-model}

    \subsection{Architecture}\label{subsec:architecture}
    \subfile{sections/model/arch.tex}

    \subsection{Optimization}\label{subsec:optimization}
    \subfile{sections/model/opt.tex}

    %%%


    \section{Experiments}\label{sec:experiments}

    \subsection{Plan}\label{subsec:plan}
    \subfile{sections/experiments/plan.tex}

    \subsection{Analysis of Gradient Descent Optimization with Neural Networks Model}
    \label{subsec:analysis-of-gradient-descent-optimization-with-neural-networks-model}

    \subsection{Optimization with Gradient Descent for Tensor-Train Model}\label{subsec:optimization-with-gradient-descent}
    \subfile{sections/experiments/tt_gd.tex}

    \subsection{Optimization with Alternating Linear Scheme for Tensor-Train Model}\label{subsec:optimization-with-alternating-linear-scheme}
    \subfile{sections/experiments/tt_als.tex}

    \subsection{Optimization with Riemannian Gradient Descent for Tensor-Train Model}\label{subsec:optimization-with-riemannian-gradient-descent}
    \subfile{sections/experiments/tt_ro.tex}

    \subsection{Results}\label{subsec:results}
    \subfile{sections/experiments/results.tex}
    \bibliography{ref}
    \appendix

    \begin{appendices}
        \section{- FTT architecture experimentation}\label{sec:on-ftt-architecture-experimentation}

        \section{- FTT optimization experimentation}\label{sec:on-ftt-optimization-experimentation}
    \end{appendices}

\end{document}